\documentclass[12pt]{article}

\usepackage[utf8]{inputenc}
\usepackage{amsmath}

\title{Concurency and Multithreading\\
       Java concurrency Constructs}
\author{Alexandru Mosoi $<$ami650@few.vu.nl$>$\\
        student number 2000903}

\begin{document}
\maketitle


\section{Counter}

\begin{table}[h!]
  \centering
  \small
  \framebox{
    \begin{tabular}{ r | c | c | c | c | c }
    \# Threads & 1 & 2 & 3 & 4 & 8 \\
    \hline
    Simple & 2000000000 & 1000046618 & 670269482 & 500093260 & 252398649 \\
    Volatile & 2000000000 & 1370615635 & 1375495038 & 1414755523 & 1259676634 \\
    Synchronized & 2000000000 & 2000000000 & 2000000000 & 2000000000 & 2000000000 \\
    \end{tabular}
  }

  \caption{Counter value for variable N and K = 2.000.000.000 on a 4 logical
  cores processor}
  \label{tbl:counter}
\end{table}


Table \ref{tbl:counter} lists value of the counter after it was incremented
\texttt{K / N} times on \texttt{N} different threads, for a total of
\texttt{K} increments. The pseudocode of one thread is :

\begin{align*}
    \texttt{for\ }   &\texttt{(long i = 0; i < K / N; i++) \{} \\
    \texttt{\ \ \ \ }&\texttt{counter.inc()} \\
    \texttt{\}\ \ \ }&\texttt{} \\
\end{align*}

\subsection{Simple}

\emph{Simple} does not use any synchronization constructs. Java
HotSpot assumes that variable is not shared accros multiple threads and
detects an optimization oportunity. The memory access into the following
pattern:

\begin{verbatim}
read counter into local-counter
increment local-counter K/N times
write local-counter to counter
\end{verbatim}

The last thread to execute \texttt{write local-counter} sets the final
value for shared counter. Since there are \texttt{N} threads and each thread
increments the counter \texttt{K/N} times the final value of the counter
will be at least \texttt{K/N}. The slight difference in the reported value
and the theoretical one is because of lazy runtime optimization of Java HotSpot.
Before the loop is optimized the value might have been incremented separately
by all threads.

\subsection{Volatile}
\emph{Volatile} declares the internal counter as \texttt{volatile}.
This will disable optimizations and the memory access pattern changes to

\begin{verbatim}
repeat
  read counter into local-counter
  increment local-counter 1 time
  write local-counter to counter
\end{verbatim}

However when considering multiple threads, for example \texttt{t1} and
\texttt{t2}, the following \emph{race}\footnote{Both threads read the
same value, increment it and then write it back} is possible because
the incrementation is not atomic:

\begin{verbatim}
t1:read counter into local-counter

t2:read counter into local-counter
t2:increment local-counter 1 time
t2:write local-counter to counter

t1:increment local-counter 1 time
t1:write local-counter to counter
\end{verbatim}

Depending on the scheduling of the threads the final value of the counter can
take any value from 1 to \texttt{K}.

\subsection{Synchronized}

\emph{Synchronized} protects the access to internal counter with
the counter's monitor so no two threads will execute the critical
section simultaneously. The memory access pattern will be:

\begin{verbatim}
t1:in critical section
t1:  read counter into local-counter
t1:  increment local-counter 1 time
t1:  write local-counter to counter

t2:in critical section
t2:  read counter into local-counter
t2:  increment local-counter 1 time
t2:  write local-counter to counter
\end{verbatim}

The final value of the counter will always be equal to \emph{K}. This
implementation of the counter works as expected.


\end{document}
